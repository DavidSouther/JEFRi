\documentclass{article}

\usepackage{jefri}

\usepackage{fullpage}
\linespread{1.3}

\begin{document}

\jefri{DJef: Distributed JEFRi}

\tableofcontents
\newpage
\linespread{1.6}

\section{Overview}
JEFRi is an entity framework to facilitate moving highly structured data
between applications and application instances. SOAP is a technique for making
remote procedure calls between a client and a server. DJEF extends and
combines the two, allowing developers to use the distributed techniques of JEFRi
while harnassing the power and flexibility of a SOAP architecture.

DJEF extends JEFRi by adding a methods field to the Entity Attributes
specification and to the context meta specification. These are methods that a
runtime will expose to its users, either as methods of an entity or the context.
Methods must be specified similarly to the WSDL descriptions in SOAP, with some
changes to recognize JEFRi entities as available types.

\method{extend}{type, proto}{Runtime}
	{Add the methods in proto to type's prototype. Will effect {\it ALL}
	instances, both current and future, of {\ilcode type}. }
	{
		\param{type}{String type to apply prototypes to.}
		\param{proto}{Object of name/function pairs to extend and override the
			prototypes of Entities in a loaded context.}
		\returns{A reference to the runtime.}
	}

\end{document}
