\documentclass{article}

\usepackage{color}
\usepackage{listings}
\lstset{ %
language=Java,                % the language of the code
basicstyle=\footnotesize,       % the size of the fonts that are used for the code
numbers=left,                   % where to put the line-numbers
numberstyle=\footnotesize,      % the size of the fonts that are used for the line-numbers
stepnumber=1,                   % the step between two line-numbers. If it's 1, each line 
                                % will be numbered
numbersep=5pt,                  % how far the line-numbers are from the code
backgroundcolor=\color{white},  % choose the background color. You must add \usepackage{color}
showspaces=false,               % show spaces adding particular underscores
showstringspaces=false,         % underline spaces within strings
showtabs=false,                 % show tabs within strings adding particular underscores
frame=single,                   % adds a frame around the code
tabsize=4,                      % sets default tabsize to 2 spaces
captionpos=b,                   % sets the caption-position to bottom
%breaklines=true,                % sets automatic line breaking
breakatwhitespace=false        % sets if automatic breaks should only happen at whitespace
}

\usepackage{fullpage}
\linespread{1.3}

\begin{document}

\title{JEFRi Integrated Query}
\maketitle
\tableofcontents
\newpage
\linespread{1.6}
JEFRi Core uses a bare skeleton object to specify data to return from queries.
While this is a concise and powerful mechanism, it does not provide the
natural language approach to data querying that SQL and the like embrace.
JEFRi Integrated Query aims to provide a set of natural-language calls to
query data sources. While the underlying implementation will be against JEFRi
contexts, the API is easily repurposed to target SQL databases, plain arrays,
or any other data source.


%JEFRi.inq().from("User")
%	.surname("Franken")
%	.




\end{document}
