\documentclass{article}

\usepackage{color}
\usepackage{listings}
\lstset{ %
language=Java,                % the language of the code
basicstyle=\footnotesize,       % the size of the fonts that are used for the code
numbers=left,                   % where to put the line-numbers
numberstyle=\footnotesize,      % the size of the fonts that are used for the line-numbers
stepnumber=1,                   % the step between two line-numbers. If it's 1, each line
                                % will be numbered
numbersep=5pt,                  % how far the line-numbers are from the code
backgroundcolor=\color{white},  % choose the background color. You must add \usepackage{color}
showspaces=false,               % show spaces adding particular underscores
showstringspaces=false,         % underline spaces within strings
showtabs=false,                 % show tabs within strings adding particular underscores
frame=single,                   % adds a frame around the code
tabsize=4,                      % sets default tabsize to 2 spaces
captionpos=b,                   % sets the caption-position to bottom
%breaklines=true,                % sets automatic line breaking
breakatwhitespace=false        % sets if automatic breaks should only happen at whitespace
}

\usepackage{fullpage}
\linespread{1.3}

\begin{document}

\title{JEFRi - JSON Entity Framework Runtime}
\maketitle
\tableofcontents
\newpage
\linespread{1.6}

\section{Overview}
JEFRi is an entity framework. The idea of an EF is to model software as close to
the problem domain as possible. JEFRi accomplishes this by specifying an Entity
Context independent of the programming language. This context describes the
entities available in the domain model, the properties of those enties, and the
attributes of entities and properties. Further, it describes the relationships
between entities. These navigation properties allow entities to act naturally
and transparently as composite types. This allows intelligent loading of related
data, without requiring an application to load an entire object graph.

Entities are analogous to Objects in OOP, and Tables in databases. They are the
building blocks of the domain, the things the domain is modeling. These could be
receipts in a Point of Sale application, Clients in Contact Management System,
or Planets in a physics model. Properties are the pieces of data that describe
an entity --- the transaction date of the receipt, or the name of the client.
Attributes provide meta-information about entities and properties. All entities
have a key attribute, that tells which property holds the entity's keys. All
properties have a type attribute, describing what data they hold. Beyond some
minimal required data, attributes can contain any additional information the
application may need. It is the responsibility of the application to use that
data --- the context will merely hold it.

The word {\sl Object} occurs rather often in this documentation. In general, it
does not mean an instance of a class. Rather, {\sl Object} follows the
Javascript meaning of a key/value store. {\sl Entity} is used closer to the idea
of an instance in classical inheritance.

\section{Keys}
Every Entity must have a key, a single unique field identifying that entity
across all instances of the runtime. To meet this requirement in a distrubuted,
heterogeneous environment, keys are version 4 (random) UUIDs.
Keys are used exclusively to refer to entities of a specific type. To maintain
unique object graphs through flattening into transactions, the key is used as an
object identifier, saving the need to handle issues of using memory references
or incrementing integers as keys.

Entity keys are immutable.

Keys should be generated by whichever runtime initially creates a new instance
of an entity. Thus, it is important that JEFRi instances provide suitably robust
random number generators. By having the key assigned by the runtime that first
creates an entity, applications eliminate the need for a central repository of
``canonical'' key information. Without a central key store, applications can be
free to handle concurrency with whatever mechanism best fits the domain model.
Further, since applications don't need to synchronize keys, significant
reductions in bandwidth are seen. The concern of conflicting keys is mitigated
by choosing a sufficiently large key domain. Further, keys must only be unique
to entity types. Thus, randomly generating Version 4 UUIDs allows 6 million
billion keys per entity type before reaching a 50\% chance of a single collision
amongst keys.

Since a type's full URN within a runtime is technically
entity://application/context/type/uuid, it is arguable that UUIDs are "Too
Much". Then again, no one needs more than 640K. If a context expects more
entities of some type than this, it would be simple to replace the key generator
implementation with something more suitably robust for that application.

\section{Relationships}
Relationships have two ends, the "From" and and the "To" end. These ends change
roles depending on which entity is accessing the relationship. The "From" end is
always the side of the relationship doing the accessing, and the "To" end is the
side being accessed. Ends have a "Type" and a "Property". The "Property" tells
which property holds the key for that end, and the "Type" property describes how
the "To" end should be accessed. Ends have two types- "has\_a" and "has\_many".
A third type, "is\_a", is a specialized "has\_a".

Relationships can be navigated by calling the appropriate get\_navprop method of
the entity. For example, if a Receipt has an associated Client, you could call
receipt.get\_client() which would return a single Client. Conversely, from the
client end, you could call client.get\_receipts() to return an array of Receipt
entities. Notice how the navigation methods are properly pluralized. This is a
result of naming the relationships "user" and "receipts" in the context- it is
not the runtime's responsibility to handle pluralities, but the context
generator's.

\section{Contexts}
JEFRi Contexts are descriptions of the entity model. Contexts are described here
as JSON objects, though can be any persistable object serialization. Contexts
are objects with two properties --- meta, describing the overall context options,
and entities, an array of entity descriptions.

A context's meta is an object containing arbitrary keyed data for the runtime's
consumption. For example, a meta could declare "Persist On Update" to be true.
In this case, the runtime could then automatically attempt to persist an entity
any time its properties were updated. The list of canonical meta attributes and
their meaning will grow organically as JEFRi is fleshed out. Otherwise, the
documentation for a particular implementation must specify the Meta attributes
it handles, as well as their meanings.

The entities array is the heart of the context. Each entry describes one entity
available. The entry is an object with five keys. Those are `name`, the canonical
name for the entity type; `key`, the property containing the entity's UUID;
properties, an array containing the properties of the entity; relationships, an
array containing descriptions of the various navigation properties; and
attributes, an object describing various metadata about the entity itself.

The name is the canonical type name for an entity. It is used extensively to
look up property definitions from several points in the application, especially
when creating new entities, persisting entities, and navigating entity
relationships. Entity names must be unique within a context. Entity names must
be valid Javascript object keys- aka,
{\tt $=\sim$ /[a-zA-Z\_\$][a-zA-Z0-9\_\$]+/}

The key is the name of the property containing the entity's key. See the
section on entity keys for more details on how keys are used in JEFRi.

Properties is an array of objects similar in layout to entities, but without keys
or relationships. Specifically, properties must define a name, a type, and an
array of attributes. The name is the Javascript-safe property identifier, and
must be unique among the properties of an entity. The type is the
platform-agnostic datatype specifier. Canonically recognized types are int,
float, and string. Attributes is an object of arbitrary key/value pairs.
Currently recognized keys are nullable, and length for string. Again, it is the
runtime implementor's responsibility to document recognized attributes and their
behavior.

Relationships is an array describing the various navigation properties available
to an entity. A relationship has four keys--- name, the identifier for the
navigation; type, one of "is\_a", "has\_a", or "has\_many"; and to and from,
describing the entities taking part in the relationship. Name must be a locally
unique, valid Javascript identifier. Further, it defines which get\_ method must
be called to navigate the relationship (specifically, entity.get\_`name`()).
Type determines what will be returned when navigating the relationship. has\_a
and is\_a will return a single entity, while has\_many will return a (possibly
empty) array of entities. Also, there is some indication that the far side of a
has\_a or has\_many relationship will probably have a has\_a or has\_many
navigation that would lead back to the entity on the from side (however, this is
not a requirement). is\_a will probably not have a reverse navigation.

Finally, the from and to properties describe where to look up the data needed
for the navigation. In all relationships, both the from and to type fields {\it
must} resolve to a valid entity name in the same context (TODO possibly change
this so there is a posisbility for cross-context navigation?). The property
fields are used slightly differently depending on what the type of the
relationship is. In a has\_a relationship, the to.property should be the same as
the to.type.key, and the navigation property will return the one to.type whose
key is equal to the from.property. The opposite is similar for has\_many.

\subsection{Persistence Stores}
Instances of JEFRi contexts need some mechanism for long-term persistence of
entity data. Persistence stores provide this by providing a flexible,
driver-based approach.
Persistence stores must provide a persist and a get
method, each taking a JEFRi transaction (see next section) and performing the
appropriate actions. Stores should also provide an is\_async method, which tells
the user whether to expect the store to return immediately, or happen in the
background with a callback.

The persist method must function atomically per transaction. It is the
responsibility of the persistence store to ensure that transactions happen
atomically. In general, it is the persistence store's responsibility to meet the
semantics specified in the section on Transactions.

The get method must handle the myriad of filter and join operations available in
the transaction specifications. See the section on transaction specifications
for the exact semantics of get requests.

\section{Transactions}
The most important, fundamental concept in computer programing is the word-level
atomic write. Since the first computers, it was impossible to read an
inconsistent word of memory- if a read and a write occur ``simultaneously'',
the read will either return the value in memory BEFORE the write, or the value
in memory AFTER the write. It will NEVER return some of the before and some of
the after bits. Of course, at a hardware level, this only works with the
smallest addressable batch of memory--- a word. Since entities are probably
always going to be bigger than a single word (UUIDs alone are 2 full words on a
64-bit machine), it is not possible to achieve atomic entity persistence in
hardware alone. Still, it would be really nice to have guaranteed atomic writes.

JEFRi transactions fill in that gap. When a runtime is ready to persist a batch
of entity updates, it builds a serialized object representation of just the
pieces that need to be updated. By using entity keys extensively, the runtime
can flatten the object graph into an object list, with guaranteed unique
references between the entities. This transaction detail can then be sent
between runtime instances, with some guarantee that that information is
consistent (generally, that guarantee will come from TCP).

JEFRi has two transaction types: get and persist. A get transaction is used to
request data from a remote JEFRi instance. A persist transaction is used to
update data at a remote JEFRi instance. Transactions are a serialized object
with two properties: meta, describing the transaction, and entities, describing
the entities to deal with.

In a GET transaction, the entities array is a template for which entities should
be returned. The return transaction will contain all entities that meet the
criteria in the transction. Specifically, the remote instance will look at the
first object in the transaction. It will return all entities of that type whose
properties match the given properties. It will join specified navigation
properties (using just `name`, not get\_name). It will do this for each entity
in the array. In other words, the individual entities specify AND conditions,
while the multiple entities specify OR conditions.

In a PERSIST transction, the entities array has the literal data that must be
persisted. The remote JEFRi runtime is responsible for making the persist atomic
and returning a transaction filling in the additional details it has about these
entities that was not included in the original transaction (eg, a JEFRi
instance may persist an entity to a runtime using DB2 as a backend, and one
property has an update trigger setting the timestamp to `now'--- the reply
transaction would include the updated value for that property).

\subsection{Transaction Specifications}
Data for a transaction should be described as close to the domain model as
possible. However, requiring only full entities would result in more verbose
transactions than necessary. To that end, the specification to a transaction
will be an array of objects describing the entities in the transaction. The
format should be similar to:

\linespread{1}
\begin{lstlisting}
[{
	type: 'User',
	surname: 'Franken',
	prop2: value2,
	cards: {}
}]
\end{lstlisting}

\linespread{1.6}
If this were a GET transaction, the reply would have an array of entities of
User and Card, with all the User entities having surname = 'Franken', prop2 =
value2, and all the Cards correctly associated with the matched Users.
Navigation property objects can specify constraints on the nav properties
returned, and can themselves declare navigation properties to return. It is not
necessary to specity a type in a navprop --- any that are will be ignored.

If this were a PERSIST transaction, the key property {\it must } be included.
The persistant store would update (or store) the values of the entity with the
properties passed in. Navigation properties should not be included in a
transaction (they will be ignored). To save nav properties, each entity should
be added individually.

\subsubsection{Gory Get Details}
To capture the full expressivity developers need in describing data, there are
several rules transactions will use to determine the entities to return from a
GET transactions.

If the property is the entity key, the value must be a valid UUID. It will be
converted to an appropriate format for the store, and the property must match
exactly. While the UUID can be transmitted as a hex string, it is recommended to
transmit it as a 16-byte integer to conserve space.

If value is an interger, properties must match exactly.

If value is a float, either it should be matched at a $2^{-8}$ threshold or it
can be the first number of an array tuple, whose second parameter specifies the
threshold precision.

If the value is a string, it will be treated as an SQL LIKE operation.
Returned entities will have properties whose values contain the string (case
sensitive).

If the value is an array, the first field in the array may specify an operation,
or be a floating point number (see above). Valid operations are any of {\tt
$<$}, {\tt $\le$}, {\tt $>$}, {\tt $\ge$}, {\tt $=$}, and {\tt REGEX}. {\tt
REGEX} treats the next value as a PCRE regular expression, and any returned
entities will have that property matching the regex. Otherwise, the operations
are as specified by the store, but generally should be numerical ordering for
numbers, and lexicographical ordering for strings.

SUM? AVERAGE? COUNT? MAX? MIN?

GROUP BY?

If the value is an array whose first parameter is not an operator or another
array, the returned entity's property will match as an IN clause, with itegers
matching exactly and string matching as described above. If the first parameter
is an array, then each element of the value's array is treated as an ANDed where
clause, with the componentes of sub arrays following the same (value) or
(operator, value) rules presented here.

Using a common SQL driver,
\linespread{1}
\begin{lstlisting}
[{
	_type: 'User',
	surname: 'Franken',
	date_of_birth: [
		['<=', 1200000000],
		['>=', 1100000000]
	],
	cards: {}
}]
\end{lstlisting}
\linespread{1.6}
becomes
\linespread{1}
\begin{lstlisting}
SELECT
	User.user_id AS user.user_id,
	User.surname AS user.surname,
	User.given_name AS user.given_name,
	User.login AS user.login,
	User.date_of_birth AS user.date_of_birth,
	Card.card_id AS card.card_id,
	Card.user_id AS card.user_id,
	Card.name AS card.name,
	Card.email AS card.email,
	Card.phone AS card.phone,
	Card.address AS card.address,
	Card.city AS card.city,
	Card.state AS card.state,
	Card.postal AS card.postal,
	Card.country AS car.country,
FROM USER
LEFT JOIN Card on User.user_id = Card.card_id
WHERE
	User.surname LIKE '%Franken%'
	AND User.date_of_birth <= 1200000000
	AND User.date_of_birth >= 1100000000
\end{lstlisting}

\linespread{1.6}
Since this is a left join and there are no constraints on Card, this query will
return a number of rows equal to the total number of matched cards. With more
than one contact card per User, some of the User data will be duplicated. The
Entity Context the store is running in will intern all the instances of the
different Users, so there will be no duplicated entity instances in memory.

\subsection{Best Persist Practices}
Persist transactions are an application's way to send changed entities to
another instance. For a database-backed entity store, these transactions could
return quickly. For a store that has to post the transaction to a remote host,
the persist might take much longer. To unify the interface, all persists are
asynchronous, and must be passed a callback. With this in mind, there are three
seperate techniques or approaches to handling a persist transaction.

The first approach is managing the transaction directly. First, the application
readies a transaction from the EntityContext by calling {\tt ec.transaction()}.
The application can then add a number of entities to the transaction via
{\tt transaction.add(entity)}. Finally, the application shold call
{\tt transaction.persist(callback)} to persist and finalize the request. Persist
can be called multiple times on a transaction; however, there is no way to
remove an entity from a transaction. The same entity can be added multiple times
without being duplicated in the final send.

The second is persisting the changes to a single entity. Every entity has a
function {\tt persist(callback)}. This function will call another function,
{\tt entity.on\_persist(transction)}, passing in the transaction that will
handle the persist request. The entity should use this to add related data to
the transaction. This technique can be combined effectively with the first
approach.

The final technique is a pair of helper methods in the EntityContext itself. The
EC maintains an internal record of what entities have been changed, and what
entities have been added. The two methods {\tt ec.persist\_new(callback)} and
{\tt ec.persist\_changes(callback)} can be used to easily persist all entities
that have been modified, but not persisted.

\subsection{Entity Accounting}
To facilitate minimizing data transfer, entities and the EntityContext keep
track of changes that occur during runtime. The EntityContext maintains an array
of new entities and an array of modified entities. These can be used to quickly
index what needs to be sent in {\tt persist\_changes} or {\tt persist\_new}.

Entities maintain accounting information regarding their status. This is stored
in two properties and a function.

{\tt \_\_new} Boolean property, set to true if this is a new entity.

{\tt \_\_modified} Object whose keys are property names that have been updated,
and those values are the value when the object was last persisted (or the
default value, if the entity is new). When an property of an entity is set, the
entity should check in its \_\_modified object for the property. If the property
is not in \_\_modified, set \_\_modified[property] to the current value of
entity[property], update entity[property], and add itself to the EntityContext's
\_modified list. If the property {\it  is} in
\_\_modified, and the value in \_\_modified is the same as the setter's value, then
the entity should set its property and remove the property from \_\_modified. If
that was the last property in \_\_modified, the entity should remove itself from
the EntityContext's modified list.

{\tt \_status} Is a function returning one of `NEW', `MODIFIED', or `PERSISTED'.

Every entity has a default {\tt persisted} event handler that clears the
modified object, clears the new flag, and removes the entity from the
EntityContext's new and modified arrays.

\section{API}
Context Methods are called on an instance of a JEFRi runtime, and include many
data access routines. Entity methods are called on instances of entities, and
include primarily navigation properties, as well as entity-specific persistence
methods. Transaction methods are called on a class which manages large GET and
PERSIST transactions. Data Store methods mirror Transaction methods, but
are generally only called from the transaction. However, there is no formal
`user-mode' constraints, so that is a recommendation, not a requirement.

\subsection{Context Methods}
	{\tt EntityContext(URI, protos) }\
		The context CTOR. Creates a new context from the json at the URI given,
		and extends the entity classes with the protos.

	{\tt definition(type) }\
		Returns the context's definition for the specified type.

	{\tt extend(type, proto) }\
		Add the methods in proto to type's prototype. Will effect {\it ALL}
		instances, both current and future, of type.

	{\tt build(type, obj) }\
		Return an non-interned instance of type with properties filled in from
		obj.

	{\tt intern(entity, updateOnIntern) }\
		Returns a canonical instance of the entity from memory. Use this to
		translate between UUIDs and pointers (or similar machine references). If
		the boolean {\tt updateOnIntern} is true, and there is a previously
		stored entity, the old entity will be updated with any non-default
		properties of the new entity.

	{\tt expand(transaction) }\
		Take a return transaction and expand the entities in it. Intern where
		possible, and call {\tt add\_`name`} and {\tt set\_`name` where}
		appropriate.

\subsection{Entity Methods}
\indent {\tt id()} \
		Return the UUID of the object, indepented of whichever property actually
		stores it.

	{\tt \_type()} \
		Returns a string of the canonical entity type.

	{\tt get\_`name`() }\
		Follow a navigation property. Returns either a single or (possibly
		empty) array with the to end of the relationship.

	{\tt set\_`name`(to\_entity) }\
		Set the has\_a navigation property. Correctly updates everyone's nav
		property ids.

	{\tt add\_`name`(to\_entity) }\
		Add to the has\_many array.

	{\tt encode(writer) }\
		Pass the entity, its properties, and its nav properties to the
		transaction writer.

\subsection{Transaction Methods}
	{\tt Transaction(entities) }
		Constructor. Entities describes the data in the transaction. See the
		secton on Transactions for details.

	{\tt add(entities) }
		Add these entities to the transaction.

	{\tt meta(attributes) }
		Specify meta attributes. Currently supported are

	{\tt get() }
		Run the transaction as a get transaction.

	{\tt persist() }
		Run the transaction as a persist transaction.

\subsection{Events}
\subsubsection{EntityContext}
{\tt saving/saved} Called before and after the transction is persisted inside
{\tt persist\_new} and {\tt persist\_changes}.

\subsubsection{Entity}
{\tt on\_persist(transaction)} called during single entity persistence.

{\tt persisting()}

{\tt persisted()}

\subsubsection{Tranasaction}
{\tt getting()}

{\tt gotten()}

{\tt persisting()}

{\tt persisted(data)}

{\tt sending()}

{\tt sent()}

{\tt continuation()}

\subsubsection{PersistStore}
{\tt getting}

{\tt gotten}

{\tt persisting}

{\tt persisted()}

\subsubsection{Order}
This is the order of events fired during a PersistStore GET request.
\linespread{1}
\begin{lstlisting}
transaction.getting()
store.getting()
store.sending()
$.ajax(fn(){
	ec.expand(data)
	store.gotten();
	transaction.gotten()
	callback();
});
store.sent();
\end{lstlisting}

\section{Reference}

\subsection{Sample Context}
\linespread{1}
\begin{lstlisting}
{"meta":{},
"entities":[
	{	"name": "User",
		 "key": "user_id",
		 "properties": [
			{	"name": "user_id",
				"type": "int",
				"attributes": {
					"primary": "true"}},
			{	"name": "surname",
				"type": "string",
				"attributes": {
					"length": "45"}},
			{	"name": "given_name",
				"type": "string",
				"attributes": {
					"nullable": "true"}},
			{	"name": "login",
				"type": "string",
				"attributes": {
					"length": "255",
					"unique": "true"}},
			{	"name": "date_of_birth",
				"type": "float",
				"attributes": {
					"nullable": "true"}}],
		 "relationships": [
			{	"name": "cards",
				"type": "has_many",
				"to": {
					"type": "Card",
					"property": "user_id",
					"vname": "user"},
				"from": {
					"type": "User",
					"property": "user_id",
					"vname": "user"}},
			{	"name": "authinfo",
				"type": "has_a",
				"to": {
					"type": "Authinfo",
					"property": "user_id",
					"vname": "user"},
				"from": {
					"type": "User",
					"property": "user_id",
					"vname": "user"}}],
		 "attributes": {
			  "vname": "users",
			  "svname": "user"}},

	{	"name": "Manager",
		 "key": "user_id",
		 "properties": [
			{	"name": "user_id",
				"type": "int",
				"attributes": {
					"primary": "true"}}],
		 "relationships": [
			{	"name": "user",
				"type": "is_a",
				"to": {
					"type": "User",
					"property": "user_id",
					"vname": "user"},
				"from": {
					"type": "Manager",
					"property": "user_id",
					"vname": "user"}}}],
		 "attributes": {
			  "vname": "managers",
			  "svname": "manager"}},

	{	"name": "Executive",
		 "key": "user_id",
		 "properties": [
			{	"name": "user_id",
				"type": "int",
				"attributes": {
					"primary": "true"}}],
		 "relationships": [
			{	"name": "user",
				"type": "is_a",
				"to": {
					"type": "User",
					"property": "user_id",
					"vname": "user"},
				"from": {
					"type": "Executive",
					"property": "user_id",
					"vname": "user"}}],
		 "attributes": {
			  "vname": "executives",
			  "svname": "executive"}},

	{	"name": "Card",
		 "key": "card_id",
		 "properties": [
			{	"name": "card_id",
				"type": "int",
				"attributes": {
					"primary": "true"}},
			{	"name": "user_id",
				"type": "int",
				"attributes": {}},
			{	"name": "name",
				"type": "string",
				"attributes": {
					"nullable": "true"}},
			{	"name": "email",
				"type": "string",
				"attributes": {
					"nullable": "true"}},
			{	"name": "phone",
				"type": "string",
				"attributes": {
					"nullable": "true"}},
			{	"name": "address",
				"type": "string",
				"attributes": {
					"nullable": "true"}},
			{	"name": "city",
				"type": "string",
				"attributes": {
					"nullable": "true"}},
			{	"name": "state",
				"type": "string",
				"attributes": {
					"nullable": "true",
					"length": "5"}},
			{	"name": "postal",
				"type": "string",
				"attributes": {
					"nullable": "true",
					"length": "10"}},
			{	"name": "country",
				"type": "string",
				"attributes": {
					"nullable": "true",
					"length": "2"}}],
		 "relationships": [
			{	"name": "user",
				"type": "has_a",
				"to": {
					"type": "User",
					"property": "user_id",
					"vname": "user"},
				"from": {
					"type": "Card",
					"property": "user_id",
					"vname": "user"}}],
		 "attributes": {
			  "vname": "cards",
			  "svname": "card"}},

	{	"name": "Authinfo",
		 "key": "authinfo_id",
		 "properties": [
			{	"name": "authinfo_id",
				"type": "int",
				"attributes": {
					"primary": "true"}},
			{	"name": "user_id",
				"type": "int",
				"attributes": {}},
			{	"name": "username",
				"type": "string",
				"attributes": {
					"length": "45"}},
			{	"name": "password",
				"type": "string",
				"attributes": {
					"length": "45"}},
			{	"name": "activated",
				"type": "string",
				"attributes": {
					"nullable": "true",
					"length": "45"}},
			{	"name": "banned",
				"type": "string",
				"attributes": {
					"nullable": "true",
					"length": "45"}}],
		 "relationships": [
			{	"name": "user",
				"type": "has_a",
				"to": {
					"type": "User",
					"property": "user_id",
					"vname": "user"},
				"from": {
					"type": "Authinfo",
					"property": "user_id",
					"vname": "user"}}],
		 "attributes": {
			  "vname": "authinfo",
			  "svname": "authinfo"}}
]}
\end{lstlisting}

\subsection{Sample GET}
\linespread{1}
\begin{lstlisting}
{
	_type: "Datapoint",
	datatype: {
		campaign: {
			campaign_id: "aac068f1-4fea-4922-8a32-7d8cf8f2923a"
		},
		options: "[boolean]",
	},
	product: {
		brand: {}
	},
	task: {
		location: {
			country: "AU"
		}
	}
}
\end{lstlisting}
\subsubsection{Translated SQL}
\begin{lstlisting}
SELECT
	datapoints.datapoint_id AS 'datapoints.datapoint_id',
	datapoints.task_id AS 'datapoints.task_id',
	datapoints.datatype_id AS 'datapoints.datatype_id',
	datapoints.value AS 'datapoints.value',
	datapoints.product_id AS 'datapoints.product_id',
	tasks.task_id AS 'tasks.task_id',
	tasks.parent_id AS 'tasks.parent_id',
	tasks.user_id AS 'tasks.user_id',
	tasks.campaign_id AS 'tasks.campaign_id',
	tasks.location_id AS 'tasks.location_id',
	tasks.title AS 'tasks.title',
	tasks.description AS 'tasks.description',
	tasks.earliest_start AS 'tasks.earliest_start',
	tasks.deadline AS 'tasks.deadline',
	tasks.estimate AS 'tasks.estimate',
	locations.location_id AS 'locations.location_id',
	locations.latitude AS 'locations.latitude',
	locations.longitude AS 'locations.longitude',
	locations.name AS 'locations.name',
	locations.address AS 'locations.address',
	locations.city AS 'locations.city',
	locations.state AS 'locations.state',
	locations.postal AS 'locations.postal',
	locations.phone AS 'locations.phone',
	datatypes.datatype_id AS 'datatypes.datatype_id',
	datatypes.campaign_id AS 'datatypes.campaign_id',
	datatypes.name AS 'datatypes.name',
	datatypes.description AS 'datatypes.description',
	datatypes.options AS 'datatypes.options',
	campaigns.campaign_id AS 'campaigns.campaign_id',
	campaigns.client_id AS 'campaigns.client_id',
	campaigns.name AS 'campaigns.name',
	campaigns.description AS 'campaigns.description',
	products.product_id AS 'products.product_id',
	products.brand_id AS 'products.brand_id',
	products.name AS 'products.name',
	products.description AS 'products.description',
	products.size AS 'products.size',
	brands.brand_id AS 'brands.brand_id',
	brands.client_id AS 'brands.client_id',
	brands.name AS 'brands.name',
	brands.description AS 'brands.description'
FROM "datapoints"
	JOIN "tasks" ON (datapoints.task_id = tasks.task_id)
	JOIN "locations" ON (tasks.location_id = locations.location_id)
	JOIN "datatypes" ON (datapoints.datatype_id = datatypes.datatype_id)
	JOIN "campaigns" ON (datatypes.campaign_id = campaigns.campaign_id)
	JOIN "products" ON (datapoints.product_id = products.product_id)
	JOIN "brands" ON (products.brand_id = brands.brand_id)
WHERE
	campaigns.campaign_id LIKE 'aac068f1-4fea-4922-8a32-7d8cf8f2923a'
	AND
	datatypes.options LIKE '%[boolean]%'
\end{lstlisting}

\end{document}
