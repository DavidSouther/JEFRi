\documentclass{article}

\usepackage{color}
\usepackage{listings}
\lstset{ %
language=Java,                % the language of the code
basicstyle=\footnotesize,       % the size of the fonts that are used for the code
numbers=left,                   % where to put the line-numbers
numberstyle=\footnotesize,      % the size of the fonts that are used for the line-numbers
stepnumber=1,                   % the step between two line-numbers. If it's 1, each line
                                % will be numbered
numbersep=5pt,                  % how far the line-numbers are from the code
backgroundcolor=\color{white},  % choose the background color. You must add \usepackage{color}
showspaces=false,               % show spaces adding particular underscores
showstringspaces=false,         % underline spaces within strings
showtabs=false,                 % show tabs within strings adding particular underscores
frame=single,                   % adds a frame around the code
tabsize=4,                      % sets default tabsize to 2 spaces
captionpos=b,                   % sets the caption-position to bottom
%breaklines=true,                % sets automatic line breaking
breakatwhitespace=false        % sets if automatic breaks should only happen at whitespace
}

\usepackage{fullpage}
\linespread{1.3}

\begin{document}

\title{JEFRi Reporting}
\maketitle
\tableofcontents
\newpage
\linespread{1.6}

\section{Overview}
JEFRi reporting is a software module to ease the creation of formatted reports
for applications storing data in JEFRi runtimes. The reporting framework makes
full use of JEFRi features, including GET transactions and DSOAP calls, while
specifying a data format similar to that used in other places in a JEFRi app.

Specify joins from smallest to largest

{
	_aggregate: {
		_type: "receipts",
		_group: "week",
		date: ['>', '2011-05-05']
	},
	field: function,
	field: function
}

Define data types.


Specify report from largest to smallest

\end{document}
